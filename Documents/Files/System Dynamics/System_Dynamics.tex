\documentclass{article}

\usepackage[utf8]{inputenc}
\usepackage{tikz}
\usepackage{circuitikz}
\usetikzlibrary{positioning,patterns,calc,quotes,angles}

\title{System Dynamics \\ Ball on a Beam}
\author{Ahmed Ibrahim Almohamed \\ Abdullah Alhamad }
\date{August 2023}


\begin{document}


\maketitle
\section{Abstract}

The Ball on Beam system is a classic example of control and dynamics problem that describes
the interaction of a ball place on a horizontal beam.
In this Study we will analyze the system into a mathematical model and a simulation.
We are focusing on understanding the interactions between the ball's position ,the beam's angle
and the external forces (e.g gravity).
Through the application of principles from physics and control theory,
we develop a mathematical model that describes the system's behavior.
By employing simulation techniques, we explore how the system responds to different control inputs and external disturbances.
Insights gained from this analysis contribute to a deeper understanding of the fundamental concepts of dynamics, control strategies, and stability in mechanical systems.
This research not only enhances theoretical knowledge but also provides valuable insights for designing and controlling similar systems in various practical applications.

\newpage
\section{System description}
The system comprises a ball placed on a beam that is tilted at an angle $\theta$.
The beam possesses a length $l$ and is capable of rotating around a defined pivot point.
The position of the ball on the beam is denoted by $x$, where $x = 0$ corresponds to the left end of the beam,
and $x = l$ corresponds to the right end.
The beam connects to a motor via a gear mechanism, where the distance from the motor's rotor to the gear arm is represented as $d$. The gear arrangement enables the motor to apply a torque to the beam, inducing its rotation about the pivot point.
Additionally, the system involves the parameter $\phi$, which signifies the gear angle.
This angle $\phi$ characterizes the orientation of the gear linked to the motor. As the motor rotates,
the gear angle $\phi$ changes, influencing the torque exerted on the beam and consequently impacting its position and tilt angle $\theta$.
The motion of the ball and the rotation of the beam are intricately linked through factors such as the motor's torque,
the gear angle $\phi$, and the distance $d$ from the rotor to the gear arm.
The system's dynamic behavior is governed by these interrelated variables,
leading to diverse motion patterns and equilibrium conditions for the ball on the tilted beam.
The comprehension and analysis of this system entail investigating the relationships among variables $l$, $x$, $d$, $\theta$, and $\phi$, as well as considering the influence of external forces like gravity and potential control inputs.
Utilizing mathematical models and principles from mechanics enables the characterization and prediction of the system's behavior across various scenarios.


\begin{figure}[h]
	\centering
	\begin{circuitikz}
		%Beam 
		\draw [fill = gray!40 ,rotate =30] (0,0) rectangle (10,0.5) node[xshift = 1.5cm] {\large Beam};
		%Length of the Beam
		\draw [<->|,rotate = 30] (0,2) --(10,2) node[midway,above] {\large l};

		%Ball
		\draw [thick, rotate = 30] (9,1) circle (0.5);
		% Ball distance
		\draw [<->|,rotate = 30] (0,1) -- (9,1) node [midway,above] {\large x};

		% Angle line , arc, and name 
		\draw [thick] (0,0) -- (10,0);
		\draw[<->] [dashed] (2,0) arc (0:30:2) node[midway,right] {\large $\theta$};

		% Gear
		\draw [fill = red!40] (7.5,1) circle (1) node [xshift = 2cm, yshift = -1.5cm] {\large Gear};

		% Gear Measurements 
		\draw (7.5,1) -- (10,1);
		\draw (7.5,1) --(9,2.5);
		\draw [<->] [dashed] (9,1) arc (0:60:1) node[midway,right] {\large $\phi$};

		% Motor Shaft
		\draw [fill = blue!40] (8,1.5) rectangle (8.1,5);

		% distance from center of Gear to Shaft

		\draw (6,3) -- (8,1.5);
		\draw (5,2.5) -- (7.5,1);
		\draw [<->| ] (6,1.9) --(6.7,2.5) node[midway,above] {\large d};


	\end{circuitikz}
	% Caption
	\caption{System Overview}
\end{figure}

\newpage
\section{Mathematical model}


\end{document}