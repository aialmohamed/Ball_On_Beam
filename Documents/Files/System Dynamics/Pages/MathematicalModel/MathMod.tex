\chapter{Mathematical Model}
\noindent Using the Lagrange equation on this model we can drive the equation of motion of the system.
In order to get the Lagrange equation we need first to find both the kinetic and potential energy of the system.
Firstly we will take a look at the Ball Energy both kinetic and potential.
The Lagrange equation describes ,that the the difference between the two energies of a certain variable as a Lagrangian $L$.
The Lagrangian equation :

\begin{equation}\label{Lagrangian}
	\begin{split}
		L = E_{\text{kinetic}} - E_{\text{potential}}
	\end{split}
\end{equation}
%%%%%%%%%%%%%%%%%%%%%%%%%%%%%%%%%%%%%%%% BALL ENERGY %%%%%%%%%%%%%%%%%%%%%%%%%%%%%%%%%%%%%%%%%%
\subsection{Ball Energy}

Starting with the kinetic Energy of the Ball,it will be the summation of the translational kinetic energy and rotation energy.

\begin{equation}
	\begin{split}
		E_{\text{kineticBall}} = E_{\text{trans}} + E_{\text{rot}}
	\end{split}
\end{equation}
The translational energy is relative to the velocity of the ball,while the rotational energy is relative to the angular velocity.
The translational is then :
\begin{equation}
	\begin{split}
		E_{\text{trans}} = \frac{1}{2}.m_{\text{ball}}.v^2
	\end{split}	
\end{equation}
And the rotational energy is :
\begin{equation}
	\begin{split}
		E_{\text{rot}} = \frac{1}{2}.I_{\text{ball}}.\omega^2
	\end{split}
\end{equation}
Therefore :
\begin{equation}\label{kineticBall}
	\begin{split}
		E_{\text{kineticBall}} =\frac{1}{2}.m_{\text{ball}}.v^2 + \frac{1}{2}.I_{\text{ball}}.\omega^2
	\end{split}
\end{equation}
Now we need to calculate the ball's potential energy, 
But because of the tilted beam we can describe the potential energy with :
\begin{equation}
	\begin{split}\label{potential}
		E_{\text{potential}} = m_{\text{ball}}.g.(p.\sin{\theta} + r.\cos{\theta}.\sin{\theta})
	\end{split}
\end{equation}

% Symbols :
\nomenclature{$m_{\text{ball}}$}{Mass of the ball}
\nomenclature{$\omega$}{angular velocity of the ball}
\nomenclature{$I_{\text{ball}}$}{Moment of inertia of the ball}
\nomenclature{$h$}{Height of the Beam}

\newpage
\subsection{Beam Energy}
\noindent In this section we will try to describe both the kinetic and potential energy of the Beam.
Starting with the kinetic energy of the Beam ,in this case the kinetic energy comes from the rotation around a fix point in the
middle.
\begin{equation}\label{kineticBeam}
	\begin{split}
		E_{\text{kineticBeam}} = \frac{1}{2}.I_{\text{Beam}}.\omega_{\text{Beam}}^2
	\end{split}	
\end{equation}
\noindent The beam has also a potential energy due to setting the beam at a height $h$.

\begin{equation}
	\begin{split}
		E_{\text{potBeam}} = M_{\text{Beam}}.g.h \\
		E_{\text{potBeam}} = M_{\text{Beam}}.g.N.\cos{\theta}
	\end{split}
\end{equation}

% Symbols
\nomenclature{$I_{\text{Beam}}$}{Moment of inertia of the Beam}
\nomenclature{$v$}{Ball's velocity on the beam}
\nomenclature{$\omega_{\text{Beam}}$}{Beam's angular velocity}
\nomenclature{$ M_{\text{Beam}}$}{Beam's mass}
\nomenclature{$N$}{Beam's height without tilting}
\subsection{Lagrangian Dynamics}
Until this point we are using the term $p$ to describe the distance of the ball on the beam.
$p$ is the polar description of the position ,which we can change into Cartesian coordinates using equation (3),(4).
With substituting (4) in (10)
\begin{equation}\label{velocity}
	\begin{split}
		E_{\text{trans}} = \frac{1}{2}.m_{\text{ball}}.((\dot{p}.\cos{\theta}-p.\theta'.\sin{\theta} - r.\theta'.\sin{\theta})^2 + \\
		(\dot{p}.\sin{\theta}+ p.\theta'\cos{\theta} - r.\theta'.\sin{\theta}) )^2
	\end{split}
\end{equation}
On the other hand we can also substitute the term $\omega$ from (6) in (11):
\begin{equation}
	\begin{split}
		E_{\text{rot}} = \dfrac{1}{2}.I_{\text{ball}}.(\dot{\theta} - \dfrac{\dot{p}}{r})^2
	\end{split}
\end{equation}
We can plug this back to \ref{kineticBall} so we can describe the kinetic energy of the ball in term of $p$ and $\theta$:
\begin{equation}
	\begin{aligned}
		E_{\text{kineticBall}} = \frac{1}{2}.m_{\text{ball}}.((\dot{p}.\cos{\theta}-p.\theta'.\sin{\theta} - r.\theta'.\sin{\theta})^2 + \\
		(\dot{p}.\sin{\theta}+ p.\theta'\cos{\theta} - r.\theta'.\sin{\theta}) )^2 + \\
		\dfrac{1}{2}.I_{\text{ball}}.(\dot{\theta} - \dfrac{\dot{p}}{r})^2
	\end{aligned}
\end{equation}
After defining the main kinetic and potential energy of the system we can set the 
equations into the Lagrangian \ref{Lagrangian} .
\begin{multline}
		L =  \frac{1}{2}.m_{\text{ball}}.((\dot{p}.\cos{\theta}-p.\theta'.\sin{\theta} - r.\theta'.\sin{\theta})^2 + \\
		(\dot{p}.\sin{\theta}+ p.\theta'\cos{\theta} - r.\theta'.\sin{\theta}) )^2 + \\
		\dfrac{1}{2}.I_{\text{ball}}.(\dot{\theta} - \dfrac{\dot{p}}{r})^2 + \frac{1}{2}.I_{\text{Beam}}.\theta'^2 - \\
		m_{\text{ball}}.g.(p.\sin{\theta} + r.\cos{\theta}.\sin{\theta})-M_{\text{Beam}}.g.N.\cos{\theta}
\end{multline}
Now this system is really complicated and driving the equations of motion is going to be hard and long ,so we can make handful of assumptions.
One of these assumptions is ,that the ball-beam system is considered to be a mass-point system which means that the ball has a radius of zero , the beam has zero moment of inertia and the term of the
beam's potential energy is neglected.
This lead to the following Lagrangian:
\begin{equation}
	\begin{split}
		L = \dfrac{1}{2}.m_{\text{ball}}.\dot{p}^2 + \dfrac{1}{2}.m_{\text{ball}}.p^2.\dot{\theta}^2 + \dfrac{1}{2}.I_{\text{Beam}}.\dot{\theta}^2 - m_{\text{ball}}.g.p.\sin{\theta}
	\end{split}
\end{equation}
%%%%%%%%%%%%%%%%%%%%%%%%%%% Lagrangian equations of Motion %%%%%%%%%%%%%%%%%%%%%%%%%%%

\newpage
\subsection{Lagrangian equations of Motion}
The Lagrangian equations of motion use to describe the how the system behave according to the terms of $p$ and $\theta$
Equations are :

\begin{equation}\label{LEM first}
	\begin{split}
		\od{}{t}(\pd{L}{p'}) - \pd{L}{p} = 0
	\end{split}
\end{equation}
And :
\begin{equation}\label{LEM second}
	\begin{split}
		\od{}{t}(\pd{L}{\theta'}) - \pd{L}{\theta} = \tau
	\end{split}	
\end{equation}
Now solving for \ref{LEM first}
\begin{equation}
	\begin{split}
	\pd{L}{p'} = 2 .\dfrac{1}{2}.m_{\text{ball}}.\dot{p} \\
	=m_{\text{ball}}.\dot{p}
	\end{split}	
\end{equation}
Then :
\begin{equation}
	\begin{split}
		\od{}{t}(\pd{L}{p}) = m_{\text{ball}}.\ddot{p}
	\end{split}
\end{equation}
And the other term :
\begin{equation}
	\begin{split}
	\pd{L}{p} = m_{\text{ball}}.p.\dot{\theta}^2 - m_{\text{ball}}.g.\sin{\theta}
	\end{split}
\end{equation}
Thus the result of the first Lagrangian equation :
\begin{equation}
	\begin{split}
		m_{\text{ball}}.\ddot{p} -  m_{\text{ball}}.p.\dot{\theta}^2 + m_{\text{ball}}.g.\sin{\theta} = 0
	\end{split}
\end{equation}
Next we solve the second equation \ref{LEM second}:
\begin{equation}
	\begin{split}
	\pd{L}{\theta'} = m_{\text{ball}}.p^2.\dot{\theta} + I_{\text{Beam}}.\dot{\theta}
	\end{split}
\end{equation}
And : 
\begin{equation}
	\begin{split}
		\od{}{t}(\pd{L}{\theta'}) = 2.m_{\text{ball}}.p.\dot{p}.\dot{\theta} + \ddot{\theta}.(I_{\text{Beam}} + m_{\text{ball}} p^2)
	\end{split}
\end{equation}
And the other term :
\begin{equation}
	\begin{split}
		\pd{L}{\theta} = -m_{\text{ball}}.g.p.\cos{\theta}
	\end{split}
\end{equation}
Thus the the result of the second Lagrangian equation:
\begin{equation}
	\begin{split}
		2.m_{\text{ball}}.p.\dot{p}.\dot{\theta} + \ddot{\theta}.(I_{\text{Beam}} + m_{\text{ball}} p^2) + m_{\text{ball}}.g.p.\cos{\theta}  = \tau
	\end{split}
\end{equation}
These two equations describe how the external torque effects the system as an input to the system, and how the distance $p$ and the angle $\theta$ are changed accordingly.