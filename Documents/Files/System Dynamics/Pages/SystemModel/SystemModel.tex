\chapter{System Model}
Considering the system with it variable ,we can tell that the system is no a linear system due to the trigonometrical and quadratic terms in the equations.
This means that we need to linearize the system around a point called the equilibrium.
In order to do that we need first to define the state-space variable.
\begin{equation}
	\vec{x} =
	\begin{bmatrix}
		x_1 \\
		x_2 \\
		x_3 \\
		x_4 \\
	\end{bmatrix}=
	\begin{bmatrix}
		p \\
		p' \\
		\theta \\
		\theta '
	\end{bmatrix}
\end{equation}
Also the input torque will be renamed :
\begin{equation}
	\begin{split}
		\tau = u
	\end{split}
\end{equation}
Before we transform the system in terms of $\vec{x}$ and $u$ we also need to define the output of the system which will be the ball's position in this case .
\begin{equation}
	\begin{split}
		\vec{y} = p = x_1
	\end{split}
\end{equation}
Now we can rewrite the equations of motion into state-space terms.
From the first Lagrangian equation \eqref{LEM first}
\begin{equation}
	\begin{split}
		m_{\text{ball}}.\dot{x_2} -  m_{\text{ball}}.x_1.x_4^2 + m_{\text{ball}}.g.\sin{x_3} = 0 \\
		\dot{x_2} =  g.\sin{x_3} - x_1.x_4^2 
	\end{split}
\end{equation}
And from the second Lagrangian equation \eqref{LEM second}
\begin{equation}
	\begin{split}
		2.m_{\text{ball}}.x_1.x_2.x_4 + \dot{x_4}.(I_{\text{Bea,}} + m_{\text{ball}} x_1^2) + m_{\text{ball}}.g.x_1.\cos{x_3}  = u \\
		\dot{x_4} = (\dfrac{1}{(I_{\text{Beam}} + m_{\text{ball}} x_1^2)} ). (u - m_{\text{ball}}.g.x_1.\cos{x_3} - 2.m_{\text{ball}}.x_1.x_2.x_4 )
	\end{split}
\end{equation} 
Now the vector $\vec{x}$ holds all the system variables in itself and we can rewrite the Lagrangian equations with it.

\begin{equation}
	\vec{\dot{x}} =
	\begin{bmatrix}
		\dot{x_1} \\
		\dot{x_2} \\
		\dot{x_3} \\
		\dot{x_4} \\
	\end{bmatrix} =
	\begin{bmatrix}
		x_2\\
		g.\sin{x_3} - x_1.x_4^2 \\
		x_4\\
		-\dfrac{m_{\text{ball}}.g.x_1.\cos{x_3} + 2.m_{\text{ball}}.x_1.x_2.x_4}{(I_{\text{Beam} } + m_{\text{ball}} x_1^2)}
	\end{bmatrix}+
	\begin{bmatrix}
		0 \\
		0\\
		0\\
		\dfrac{1}{(I_{\text{Beam}} + m_{\text{ball}} x_1^2)}
	\end{bmatrix}*u
\end{equation}
%%%%%%%%%%%%%%%%%%%%%%%%%%%%%%%%%%%% Linearization of the State-Space %%%%%%%%%%%%%%%%%%%%%%%
\newpage
\subsection{Linearization of the State-Space}
In order to set the equilibrium point(s) we must be careful about the $x_1$ which represent the position (here :initial) of the ball.
Setting this initial value to 0 will cause the equations to draft to infinity ,which is not possible.
This means that we need to chose an equilibrium where all the other values except $x_1$ is set to zero, while $x_1$ is set $x_{\text{eq}}$.
In other word , The Ball has no initial velocity , the beam's angle is zero and has no angular velocity but the ball is positioned at $x_{\text{eq}}$ unit of measurement.
where $x_{\text{eq}} \epsilon R\backslash\{0\}$ :

\begin{equation}
	\vec{x_*} =
	\begin{bmatrix}
		x_{\text{$1_*$}} \\
		x_{\text{$2_*$}} \\
		x_{\text{$3_*$}} \\
		x_{\text{$4_*$}}
	\end{bmatrix}=
	\begin{bmatrix}
		x_{\text{eq}}\\
		0 \\
		0 \\
		0 \\
	\end{bmatrix}
\end{equation}
We need also to find equilibrium point of the External torque $u$ which we will to some torque $T$.

\begin{equation}
	\begin{split}
		u_* = T
	\end{split}
\end{equation}
Further to do, is to find the jacobian matrix around the equilibrium point(s) ,by doing this we will reach a linearized state-space model of the system,
which we can test ,simulate and find more about the project before we start the implementation.
Also to reach the normal form of the State-space which is :

\begin{equation}
	\begin{split}
		\vec{\dot{x}} = A*\vec{x} + B*\vec{u};\\
		\vec{y}  = C*\vec{x} + D*\vec{u}
	\end{split}
\end{equation}
The jacobian will give us the linearized matrices $\Delta A$,$\Delta B$,$\Delta C$ and $\Delta D$.
\begin{equation}
	\begin{split}
		\Delta A = \pd{\vec{\dot{x}}}{\vec{x}}\|x_*,u_*\\
		\Delta B = \pd{\vec{\dot{x}}}{\vec{u}}\|x_*,u_*\\
		\Delta C = \pd{\vec{y}}{\vec{x}}\|x_*,u_* \\
		\Delta D = \pd{\vec{y}}{\vec{u}}\|x_*,u_*
	\end{split}
\end{equation}
Now we need to find the value of these matrices.
Starting with $\Delta A$ evaluating it at $\vec{x_*}$ and $u_*$
\begin{equation}
	\Delta A =
	\begin{bmatrix}
		0 &1 &0 &0\\
		0 &0 & g &0\\
		0 &0 &0 &1 \\
		\gamma_1 & 0 & 0 & 0
	\end{bmatrix}
\end{equation}
Where $\gamma_1 = \dfrac{g.m_{\text{ball}}^2.x_{\text{eq}}^2 + 2.T.x_{\text{eq}}- I_{\text{Beam}}.g}{(m_{\text{ball}}.x_{\text{eq}}^2 + I_{\text{Beam}})^2}$
\newpage
Then $\Delta B$ is :

\begin{equation}
	\Delta B =
	\begin{bmatrix}
		0 \\
		0 \\
		0 \\
		\gamma_2
	\end{bmatrix}
\end{equation}
Where $ \gamma_2 = - \dfrac{1}{m_{\text{ball}}.x_{\text{eq}}^2 + I_{\text{Beam}}}$

To Get $\Delta C $
\begin{equation}
	\Delta C =
	\begin{bmatrix}
		1 & 0 & 0 & 0
	\end{bmatrix}
\end{equation}
And when calculate $\Delta D $ ,,we found that simply it is zero ,because there is no $u$ term in the $\vec{y}$ equation.
Therefore :
\begin{equation}
	\begin{split}
		\Delta D = 0
	\end{split}
\end{equation}